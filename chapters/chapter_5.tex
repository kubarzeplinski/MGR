\chapter{Wymagania funkcjonalne i niefunkcjonalne}
Jak już zostało wspomniane w rozdziale dotyczącym serwisu Twitter, przy okazji omówienia dostępnych narzędzi, nie ma obecnie na rynku aplikacji, która umożliwiałaby śledzenie występowania dowolnego słowa w wiadomościach zamieszczanych w czasie rzeczywistym w tym serwisie, rysowałaby wykres zależności pomiędzy użytkownikami, analizowałaby nastroje społeczne użytkowników z wyświetleniem informacji o nastroju wyrażanym w poszczególnych wiadomościach oraz pozwalałaby na analizowanie danych historycznych i zamieszczanych w czasie rzeczywistym. Głównym celem tej pracy dyplomowej jest stworzenie aplikacji, która posiadałaby wspomniane funkcjonalności.

\section{Wymagania funkcjonalne}
Główną funkcjonalnością, którą powinna zapewnić budowana aplikacja jest dostęp do usystematyzowanych danych pochodzących z sewisu Twitter, które będą nieść ze sobą informację o sentymencie. Dane te powinny być także przechowywane w taki sposób, aby móc zapewnić do nich dostęp w dowolnym momencie oraz spoza zaimplementowanego narzędzia. \\
Wymagania funkcjonalne, które dotyczą budowanego rozwiązania to: 
\begin{itemize}
	\item[--] \textbf{przetwarzanie danych z serwisu Twitter} - aplikacja powinna przetwarzać tweety użytkowników serwisu Twitter, do których dostęp można uzyskać przez Streaming API tego serwisu, które zostało szczegółowo omówione w rozdziale 2.; 
	\item[--] \textbf{filtrowanie napływających danych po słowie kluczowym} - dane napływające w czasie rzeczywistym powinny być filtrowane pod względem zawartości w treści wiadomości słowa kluczowego, które zostało określone przez użytkownika za pomocą graficznego interfejsu aplikacji;
	\item[--] \textbf{zapis napływających danych} - budowane narzędzie powinno zapisywać przetworzone, uporządkowane i dotyczące wybranego słowa kluczowego dane w lokalnej bazie danych, która umożliwiałaby dostęp do nich przez swój wbudowany pulpit w dowolnym momencie oraz spoza zaimplementowanego narzędzia;
	\item[--] \textbf{duża częstotliwość pobierania danych} - aplikacja powinna pobierać informacje w krótkich odstępach czasu, ponieważ serwis Twitter charakteryzuje duża ilość informacji przesyłanych w każdej sekundzie;
	\item[--] \textbf{prezentowanie danych historycznych} - narzędzie powinno prezentować dane historyczne zgromadzone podczas przetwarzania danych napływających wówczas w czasie rzeczywistym;
	\item[--] \textbf{prezentowanie podstawowych danych napływających w czasie rzeczywistym} - aplikacja powinna umożliwiać analizowanie podstawowych informacji o wiadomościach i użytkownikach, które będą napływać w czasie rzeczywistym;
	\item[--] \textbf{dostęp do szczegółowej informacji o użytkowniku} - implementowane narzędzie powinno umożliwiać dostęp do informacji o każdym użytkowniku zainteresowanym wybranym słowem kluczowym, którego wiadomość udało się zarejestrować podczas gromadzenia danych z wykorzystaniem Streaming API serwisu Twitter;
	\item[--] \textbf{dostęp do szczegółowej informacji o wiadomości} - aplikacja powinna wyświetlać szczegółową informację, o każdej wiadomości zawierającej wybrane słowo kluczowe i zapisanej podczas gromadzenia danych z wykorzystaniem Streaming API;
	\item[--] \textbf{prezentowanie informacji statystycznej o sentymencie na dany temat} - narzędzie powinno prezentować statystyki sentymentu użytkowników serwisu Twitter, którzy w swoich wiadomościach zawarli wybrane słowo kluczowe i których wiadomości udało się zapisać podczas analizy danych napływających w czasie rzeczywistym;
	\item[--] \textbf{możliwość jednoczesnej analizy danych historycznych i napływających w czasie rzeczywistym} - aplikacja powinna umożliwiać jednoczesną analizę danych historycznych i napływających w czasie rzeczywistym bez konieczności ponownego uruchamiania aplikacji lub zatrzymywania jednej z analiz;
	\item[--] \textbf{przyjazny interfejs graficzny} - narzędzie powinno posiadać wygodny, łatwy do nauczenia oraz prosty interfejs graficzny.
\end{itemize}


\section{Wymagania niefunkcjonalne}
Tworzone rozwiązanie będzie także analizowało wydźwięk wypowiedzi użytkowników. Wymagania niefunkcjonalne, które powinna spełniać przygotowywana aplikacja to: \\
\\
\textbf{Spełnienie głównych założeń systemu czasu rzeczywistego} \\
\\
Głównym przypadkiem biznesowym, dla którego tworzona jest wspomniana aplikacja jest sytuacja dużego i globalnego zainteresowania pewnym tematem, które objawia się odnoszeniem się do niego w wiadomościach zamieszczanych przez użytkowników serwisu Twitter. Można stwierdzić, że przygotowywane narzędzie będzie przykładem systemem czasu rzeczywistego jeśli system ten będzie "\textit{urządzeniem techniczne, którego wynik i efekt działania będzie zależny od chwili wypracowania tego wyniku}". Wspólną cechą definicji takiego systemu jest "\textit{zwrócenie uwagi na równoległość w czasie zmian w środowisku oraz obliczeń realizowanych na podstawie stanu środowiska}". \\
\\
\textbf{Płynna obsługa danych} \\
\\
Budowany system powinien przetwarzać dane w czasie nie większym niż tempo napływania nowych informacji. Koniecznością jest zatem skorzystanie z narzędzi umożliwiających sprawne przetwarzanie danych, ale także ich zapis do bazy w czasie nie większym niż czas trwania określonego okna czasowego. \\
\\
\textbf{Możliwość działania aplikacji i analizowania danych historycznych w trybie offline} \\
\\
Dane zapisane podczas sesji korzystania ze strumienia danych serwisu Twitter będą zapisywane w lokalnej bazie danych, dlatego stworzone narzędzie powinno umożliwiać analizowanie danych historycznych bez połączenia z internetem. \\
\\
\textbf{Jednorazowe przetwarzanie informacji ze strumienia danych} \\ 
\\
Aplikacja powinna tylko jeden raz przetwarzać i zapisywać do bazy danych informacje pozyskane ze strumienia. Jeśli w bazie istnieje już informacja o użytkowniku to nowe wiadomości powinny być z nim powiązane. \\ 
\\ 
\textbf{Niezawodność} \\
System powinien charakteryzować się niezawodnością podczas pracy z dużą ilością danych napływających w krótkich odstępach czasu oraz jak najbliższym prawdy określaniem sentymentu wypowiedzi zawartej w wielu wiadomościach.

\section{Podsumowanie}
System napisany na potrzeby tej pracy dyplomowej powinien spełniać wszystkie z wymienionych wymagań niefunkcjonalnych, ponieważ nie spełnienie nawet jednej z nich może spowodować, że aplikacja będzie nieużyteczna. Postawione wymagania funkcjonalne i niefunkcjonalne, łącząc się z opisem serwisu Twitter, narzędzi Big Data oraz przetwarzania języka naturalnego, definiują potrzeby jakie powinny umożliwiać narzędzia wybrane do jej budowy oraz samo narzędzie. Zostanie to omówione w następnych rozdziałach.

