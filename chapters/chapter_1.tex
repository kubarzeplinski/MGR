\chapter{Wstęp}
\qquad Czynnikiem powodującym rozwój świata są informacje i umiejętne ich wykorzystanie. Od jakiegoś czasu źródłem informacji stał się internet, a szczególnie portale społecznościowe, gdzie ludzie wymieniają się informacjami na różne tematy. Wiele globalnych instytucji korzysta z tego typu serwisów, wydając za ich pośrednictwem oficjalne komunikaty. Ilość informacji generowanych przez użytkowników takich serwisów jest tak duża, że wymaga użycia specjalnych narzędzi \textit{Big Data}. Jednak decydującym czynnikiem jest umiejętne wykorzystanie wiedzy zawartej w zgromadzonych informacjach, do czego potrzebny jest czynnik ludzki.

Jednym z ciekawych sposobów wykorzystania informacji z serwisów takich jak np. \textit{Twitter} jest badanie wydźwięku wpisów zamieszczanych przez użytkowników zwanego od angielskiego sformułowania \textit{sentiment analysis} sentymentem. Aby było to możliwe potrzebne jest przetworzenie języka, którym posługują się ludzie czyli \textit{języka naturalnego} na postać, którą mogą posługiwać się systemy NLP (ang. \textit{Natural Language Processing}), a następnie określenie sentymentu za pomocą podejścia słownikowego lub technik maszynowego uczenia się (ang. \textit{machine learning}). Wykorzystanie takich informacji pozwala zbadać opinie ludzi na praktycznie dowolny temat.

\textbf{Cel pracy}

Celem niniejszej pracy jest przedstawienie możliwośći wykorzystania analizy sentymentu wypowiedzi użytkowników serwisu społecznościowego Twitter. Stworzone powinno zostać rozwiązanie umożliwiające przetwarzanie w czasie rzeczywistym wiadomości publikowanych w tym serwisie zawierających wybrane słowo kluczowe. Napisany system powinien wyświetlać statystyki wydźwięku opinii użytkowników tego serwisu.

\textbf{Zakres pracy}

Pracę dyplomową można podzielić na trzy części: przegląd najistotniejszych informacji o poruszanych tematach, opis napisanej aplikacji oraz omówienie przeprowadzonych badań i wyciągnięte z nich wnioski.

W pierwszej części w rozdziale 2. opisano zasadę działania serwisu społecznościowego Twitter wraz z udostępnionym przez niego programowalnym interfejsem (ang. API - \textit{Application Programming Interface}). W rozdziale 3. zostały podane informacje na temat analizy sentymentu wypowiedzi w systemach komputerowych. 

W drugiej części w rozdziale 4. przedstawiono wymagania funkcjonalne i niefunkcjonalne postawione przed zbudowaną aplikacją. W rozdziale 5. opisano wybrane do jej implementacji narzędzia, biblioteki oraz języki programowania. W rozdziale 6. znajduje się omówienie stworzonego systemu \textit{Twitter Analyser}.

W ostatniej części w rozdziale 7. omówiono badania sentymentu wypowiedzi przeprowadzone podczas dwóch wydarzeń - meczu piłki nożnej FC Barcelona - Real Madryt oraz Święta Dziękczynienia. W ostatnim rozdziale 8. znajduje się podsumowanie całej pracy dyplomowej.