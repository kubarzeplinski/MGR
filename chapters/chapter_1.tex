\chapter{Wstęp}
Czynnikiem powodującym rozwój świata są informacje i umiejętne ich wykorzystanie. Od jakiegoś czasu źródłem informacji stał się internet, a szczególnie portale społecznościowe, gdzie ludzie wymieniają się informacjami na różne tematy. To na serwisach tego typu często posiadają konta ważne globalne instytucje i stało się normalne, że za ich pomocą wydają oficjalne komunikaty. Ilość informacji generowanych przez użytkowników takich serwisów jest tak duża, że wymaga użycia specjalnych narzędzi \textit{Big Data}. Jednak decydującym czynnikiem jest umiejętne wykorzystanie wiedzy zawartej w zgromadzonych informacjach, do czego potrzebny jest czynnik ludzki. \\
Jednym z ciekawych sposobów wykorzystania informacji z serwisów takich jak np. \textit{Twitter} jest badanie wydźwięku wpisów zamieszczanych przez użytkowników zwanego sentymentem od angielskiego sformułowania \textit{sentiment analysis}. Aby było to możliwe potrzebne jest przetworzenie języka, którym posługują się ludzie czyli \textit{języka naturalnego} na postać, którą mogą posługiwać się systemy NLP (ang. \textit{Natural Language Processing}), a następnie określenie sentymentu za pomocą podejścia słownikowego lub technik maszynowego uczenia się (ang. \textit{machine learning}). Wykorzystanie takich informacji pozwala zbadać opinie ludzi na praktycznie dowolny temat. \\
Przedstawiona praca dyplomowa opisuje także aplikację internetową stworzoną na potrzeby niniejszej pracy, która spełnia wymagania systemu czasu rzeczywistego - przetwarza i analizuje wiadomości przychodzące w czasie rzeczywistym z serwisu Twitter zawierające podane słowo kluczowe oraz bada wydźwięk opinii użytkowników tego serwisu.