\chapter{Serwis społecznościowy Twitter}
Serwis spo\l{}eczno\'sciowy Twitter jest ameryka\'nskim serwisem internetowym s\l{}u\.z\k{a}cym g\l{}\'ownie do zamieszczania wiadomo\'sci tzw. "tweet\'ow", kt\'ore u\.zytkownicy tego serwisu mog\k{a} komentowa\'c lub przekazywa\'c dalej. Tweety by\l{}y pocz\k{a}tkowo ograniczone do 140 znak\'ow, ale limit ten zosta\l{} podwojony w 2017 r. dla wszystkich j\k{e}zyk\'ow opr\'ocz chi\'nskiego, japo\'nskiego i korea\'nskiego. Serwis ten, nazywany SMS internetu, zosta\l{} za\l{}o\.zony w 2006 r. i sukcesywnie zwi\k{e}ksza\l{} swoj\k{a} popularno\'s\'c poprzez wzrost liczby u\.zytkownik\'ow odwiedzaj\k{a}cych jego witryn\k{e} oraz wysy\l{}aj\k{a}cych wiadomo\'sci. W 2012 r. osiągnął‚ ponad 100 milionów użytkowników, którzy zamieszczali łącznie ponad 340 milionów wiadomo\'sci dziennie wraz z obsługą \'srednio około 1.6 miliarda wyszukujących zapytań dziennie. W 2013 r. Twitter stał si\k{e} jedną z najcze\'sciej odwiedzanych stron w całym internecie. Na początku 2016 r. serwis ten posiadał ponad 319 milionów użytkowników aktywnych podczas każdego miesiąca. Od kilku lat Twitter stał si\k{e} serwisem gdzie dochodzi do wymiany zdań na różny temat dotyczących np. polityki, sportu, produktów, wydarzeń społecznych.