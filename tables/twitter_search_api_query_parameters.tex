\begin{table}
\centering
\caption{Parametry żądania Twitter Search API}
\label{tab:table1}
\begin{tabularx}{\linewidth}{|c|c|X|c|}\toprule
    Parametr & {Wymagany/Opcjonalny} & {Opis} & {Przykład} \\ \midrule
    q & wymagany & {zapytanie wyszukujące o maksymalnej długości 500 znaków} & nasa \\ \midrule
    geocode & opcjonalny & {zwraca wiadomości użytkowników oddalonych o podany promień od podanej szerokości i długości geograficznej, promień może być podany w milach lub kilometrach} & {37.781157 -122.398720 1mi} \\ \midrule
    lang & opcjonalny & {ogranicza wiadomości do wybranego języka spośród dostępnych kodów ISO 639-1} & {pl} \\ \midrule
    locale & opcjonalny & {specyfikuje język wysyłanego zapytania, obecnie tylko \textit ja jest skuteczny} & {ja} \\ \midrule
    result\textunderscore type & opcjonalny & {określa typ zwracanych wiadomości, obecnie dostępne są trzy wartości tego parametru: \textit{recent} (zwracane są najnowsze wiadomości), \textit{popular} (zwracane są najbardziej popularne wiadomości) \textit{mixed} (wartość domyślna, zwracane wyniki obejmują najnowsze i najbardziej popularne wiadomości)} & {mixed} \\ \midrule
    count & opcjonalny & {specyfikuje ilość zwracanych wiadomości; maksymalna wartość to 100, a domyślna to 15} & 100 \\ \midrule
\end{tabularx}
\end{table}
