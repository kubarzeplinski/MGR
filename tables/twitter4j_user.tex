\begin{table}
\centering
\caption{Dokumentacja interfejsu User z pakietu twitter4j [6]. Zamieszczone zostały tylko najważniejsze z metod.}
\label{tab:table1}
\begin{tabularx}{\linewidth}{|c|c|X|}\toprule
    Typ zwracany & Nazwa metody & Opis \\ \midrule
    java.util.Date & getCreatedAt() & zwraca datę utworzenia profilu użytkownika \\ \midrule
    java.lang.String & getDescription() & zwraca opis konta użytkownika \\ \midrule
    java.lang.String & getEmail() & zwraca adres e-mail powiązany z tym kontem \\ \midrule
    int & getFavouritesCount() & zwraca liczbę wiadomości, którą polubił ten użytkownik \\ \midrule
    int & getFollowersCount() & podaje ilość użytkowników śledzących profil \\ \midrule
    int & getFriendsCount() & podaje ilość śledzonych profili \\ \midrule
    long & getId() & zwraca id użytkownika \\ \midrule
    java.lang.String & getLang() & zwraca język preferowany przez użytkownika \\ \midrule
    java.lang.String & getLocation() & zwraca lokalizację użytkownika \\ \midrule
    java.lang.String & getName() & podaje nazwę użytkownika \\ \midrule
    java.lang.String & getScreenName() & zwraca nazwę konta \\ \midrule
    Status & getStatus() & zwraca obiekt typu Status reprezentujący wiadomość wysłaną przez użytkownika \\ \midrule
    int & getStatusesCount() & podaje ilość wiadomości wysłanych przez użytkownika \\ \midrule
    java.lang.String & getTimeZone() & podaje strefę czasową użytkownika \\ \midrule
    java.lang.String & getURL() & zwraca URL do profilu \\ \midrule
    boolean & isVerified() & podaje informację czy profil jest zweryfikowany \\ \bottomrule 
\end{tabularx}
\end{table}
